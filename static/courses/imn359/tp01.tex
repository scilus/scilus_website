\documentclass{article}
%  File:	tp01.tex
%  Created:	Apr 24, 2010
%  Author: 	Maxime Descoteaux

%%%%%%%%%%%%% Definitions %%%%%%%%%%%%%%%%%%

%%%% Largeur et tailles des marges
\oddsidemargin 0in
\evensidemargin 0in
\textwidth 6.5in

%%%%% Pour les figures, graphiques et math (amssymb)
\usepackage{color,graphicx,float,epsfig,amssymb}

%%% Pour avoir les accents %%%%%%%%%%%%%%%%%
\usepackage[utf8]{inputenc}
\usepackage[french]{babel}
%%%%%%%%%%


%%%%%%%%%%%%% Document %%%%%%%%%%%%%%%%%%%%%
\title{TP 1 IMN359 - Rappels mathématiques \\
(Nombres complexes, algèbre linéaire, produit Hermitien)}
\author{Maxime Descoteaux}
\date{\today}
\begin{document}
\maketitle

Ce TP 1 est à me remettre par courriel {\bf le 14 septembre} 
dans un seul fichier zip bien organisé. Le but du TP est de vous
familiariser avec Latex et Python tout en faisant un rappel
mathématique. Même si vous travaillez en équipe, essayez de faire
toutes les questions seul(e)s et de comprendre le Latex. 
Vous devez rédiger un rapport avec les solutions en 
\emph{Latex} et me remettre un zip avec votre code Python. Commentez
le code et assurez-vous que je puisse reproduire vos 
résultats. {\bf Ne PAS utilisez la toolbox symbolique de Python,
  i.e. pas besoin de coder le calcul des dérivées. Vous le faites à la
main et vous implémentez de façon numérique.}
Séparez votre code en différents fichiers pour faciliter la
lecture. Des points seront attribués pour la qualité du document latex
(5 points) et la qualité du code Python (5 points). Le TP vaut 40
points au total. Si vous faites des questions à plus qu'une équipe,
dites-le sur le rapport. 

\vspace{1cm}
\begin{enumerate}

\item {\bf Racines carrés et forme d'Euler [5 points]}
  \begin{enumerate}
    \item Trouvez les racines de $x^2 + 2x + 5$.
      % $x^2 + 4x + 5$.

    \item Ecrivez vos racines sous la forme d'Euler.

    \end{enumerate}
\vspace{1cm}
\item {\bf Séries de Taylor [10 points]}
  \begin{enumerate}
    \item Trouvez la série de Taylor de $\displaystyle \frac{1}{1 -
    x}$, autour de $a = 0$. Décrivez vos étapes.


    \item Ecrivez une fonction Python qui implémente cette
      série. Testez votre fonction pour 3 différentes valeurs de $x$
      avec l'ordre de votre choix. 

    \item En python, faites une courbe qui illustre l'erreur de la série en
      fonction de l'ordre auquel la série est tronquée pour la
      fonction évaluée à $x = 0.5$. 

    \item A l'aide des séries de Taylor, démontrez que 
       \sin(\theta) = \frac{e^{i\theta} - e^{-i\theta}}{2i}
       $$                  
  \end{enumerate}
  
\vspace{1cm}
\item {\bf Systèmes d'équations. [10 points]} Soit le système suivant:
$$
\begin{array}{ccccccc}
2x & - & y  &   &    & = &  0 \\
-x & + & 2y & - & z  & = & -1 \\
   & - & 3y & + & 4z & = &  4
\end{array}
$$
\begin{enumerate}
 \item Dessinez une représentation géométrique de ce système. Vous
   pouvez prendre le logiciel que vous voulez. Suggestion: je vous
   suggère de regarder matplotlib et les fonctions suivantes en ligne.
\begin{verbatim}
        import matplotlib.pyplot as plt 
        from mpl_toolkits.mplot3d import Axes3D
\end{verbatim}
 \item Trouvez la solution $(x,y,z)$ avec la méthode 
   classique de substitution.
 \item Ecrivez le système sous forme matricielle $A x = B$.
 \item Vérifiez votre solution en Python.
\end{enumerate}

\vspace{1cm}

\item {\bf Bases orthogonales et orthonormales. [5 points]} 
  \begin{enumerate}
    \item 
      $\overrightarrow{z}^1 = (1 - i, 1)$ et
      $\overrightarrow{z}^2 = (i, 1-i)$ $\in \mathcal{C}^2$ 
      (nombres complexes dans le
      plan 2D). Montrez que $\overrightarrow{z}^1$ et $\overrightarrow{z}^2$
      sont orthogonaux. 

  \item Est-ce qu'ils sont orthonormés? Pourquoi?
    
  \item Donnez une base orthonormale qui définie  l'espace
    $\mathcal{C}^2$.

  \end{enumerate}


\end{enumerate}

\end{document}
