\documentclass{article}
%  File:	tp02.tex
%  Created:	Sep 16, 2010
%  Author: 	Maxime Descoteaux

%%%%%%%%%%%%% Definitions %%%%%%%%%%%%%%%%%%

%%%% Largeur et tailles des marges
\oddsidemargin 0in
\evensidemargin 0in
\textwidth 6.5in


%%% Pour avoir les accents %%%%%%%%%%%%%%%%%
%\usepackage[latin1]{inputenc}
%\usepackage[francais]{babel}
\usepackage[utf8]{inputenc}
\usepackage[french]{babel}
%%%%%%%%%%


%%%%%%%%%%%%% Document %%%%%%%%%%%%%%%%%%%%%
\title{TP2 - IMN359 - Série de Fourier et Transformée de Fourier}
\author{Maxime Descoteaux}
\date{\today}
\begin{document}
\maketitle

Ce TP 2 est à me remettre par courriel à la date déterminer en classe. 
Vous devez rédiger un rapport avec les solutions en 
\emph{Latex} et me remettre un zip avec votre code python. Commentez
le code et assurez-vous que je puisse reproduire vos 
résultats et figures. 
Séparez votre code en différents fichiers pour faciliter la
lecture. Des points seront attribués pour la qualité du document latex
et ses figures (5 points), et la qualité du code python (5 points). 

\section*{}
\begin{enumerate}

\item {\bf Produit Hermitien. [10 points]} Soit, 
  $f(x) = \cos(x)$ et $g(x) = \cos(4x)$,
  calculez et démontrez le résultat de l'intégrale suivante: 
  $$
  \int_{-\pi}^{\pi} f(x) g(x) dx.
  $$
  Que concluez-vous sur les fonctions $f(x)$ et $g(x)$?
  \\
  \\
  Vérifiez votre résulat en python
\vspace{1cm}


\item  {\bf Périodes [10 pts]}
  \begin{enumerate}
    \vspace{0.5cm}
  \item Quelle est la période fondamentale de $\sin^2(t)$? Montrez votre
    démarche. Illustrez votre résultat en python.
    \vspace{1cm}    
  \item Quelle est la période fondamentale de 
    $$
    f(t) = 2\cos(t) +    \cos(t/3) + 3\cos(t/5).  
    $$
    En vous inspirant des notes de cours, illustrez
    les périodes de chacun des cosinus et illustrez la période
    fondamentale. Faites un beau graphe représentant les différents
    cosinus et f(t). Mettez-le dans le latex.

\end{enumerate}

\newpage 

\item {\bf Série de Fourier [35 pts]}
  \begin{enumerate}
    \vspace{0.5cm}
  \item Déterminer les coefficients de la série de Fourier réelle (SF)
    de $f(t) = \Lambda(t)$ de période 2. % avec T = 2
    $$
    \Lambda(t) = \left \{
      \begin{array}{cl}
        1 - |t| &   |t| < 1 \\
        0 & |t| > 1
      \end{array} \right .
    $$  
    \vspace{1cm}
  \item Ecrivez la SF réelle de $f(t)$. Prenez la période de 2 entre
    [-1, 1]. 
    \vspace{1cm}    
  \item Ecrivez la version complexe de
    la SF de $f(t)$. 
    \vspace{1cm}    
  \item Sur un intervalle de [-20, 20], faites le graphe
    de $\Lambda(t)$ et de ses SF réelle et complexe en utilisant $100$
    harmoniques. Mettez le graphe dans le latex. 
    \vspace{1cm}    
  \item Calculez l'approximation de la SF complexe
    faite pour 3 différents nombres d'harmoniques. Calculez
    l'erreur quadratique moyenne numériquement et analytiquement de la
    SF complexe pour chacune des approximations. 
    \vspace{1cm}
  \item Trouvez et illustrez la SF complexe de f(t) de période 20 suivante:  
      $$
      f(t) = \left \{
        \begin{array}{cl}
          1   & |t| < 5 \\
          0   & 5 < |t| < 15 \\
          1   & |t| > 15
        \end{array} \right .
      $$  
      \vspace{1cm}
    \item Evaluez la SF de f(t) aux points t = 5, t = 10, t =
      15. \\
      (indice: vous devriez me parler des conditions de Dirichlet)
      
    \end{enumerate}
\vspace{1cm}

\newpage

\item {\bf Transformée de Fourier (TF) [10 pts]}
  \begin{enumerate}
    \vspace{0.5cm}

%   \item Calculez la TF de $g(t) = a \Pi( \frac{t - b}{c})$. En faisant un subplot,
%     illustrez g(t), la partie réelle et imaginaire de sa TF, son spectre
%     de phase et son spectre d'amplitude. Mettez-le dans le latex.
    
% $$
% \Pi(t) = \left \{ 
%   \begin{array}{cl}
%     1 & -1/2 \leq t \leq 1/2, \\
%     0 & \textrm{sinon}
%     \end{array}
%   \right .
%   $$
  
  \item Calculez la TF de $g(t) = \delta(2t + 1)$
  
    \vspace{1cm}

  %% \item Calculez la TF de $g(t) = \displaystyle \frac{\cos(\pi t)}{\pi(t-1/2)}$. En
  %%   faisant un subplot, 
  %%   illustrez g(t), la partie réelle et imaginaire de sa TF, son spectre
  %%   de phase ($\theta$) et son spectre d'amplitude (avec le
  %%   $abs$). Mettez-le dans le latex. 

    \vspace{1cm}

  \item Sachant que la transformée de Fourier (TF) de la fonction
    porte $\Pi(t)$ est le sinus cardinal $\textrm{sinc}(w / 2)$, c-a-d
    $\textrm{TF}[\Pi(t)] = \textrm{sinc}(w/2)$, 
    utilisez les propriétés de la Table 8.8 des notes de cours
    pour résoudre les TF suivantes. 
    Ecrivez quelle(s) propriété(s) vous utilisez.
   \vspace{0.5cm}
    \begin{enumerate}
    \item $\Pi( \frac{t-1}{3} )$
   \vspace{0.5cm}
    \item  $\textrm{sinc}(t) + \textrm{sinc}(3t)$
    \vspace{1cm}
    \end{enumerate}

    \vspace{1cm}

\end{enumerate}

\item {\bf Transformée de Fourier (TF) 2D [10 pts]}. \\
  Calculez et démontrez la
  transformée de Fourier 2D des 
  fonctions suivantes. Prenez les propriétés que vous voulez et
  décrivez ce que vous faites.  
  \vspace{1cm}    
  \begin{enumerate}
  \item  %$f(x_1, x_2) = \mathbf{1}_{[0,1]^2}(x_1, x_2)$
    $f(x_1, x_2) = \Pi( x_1 + \frac{1}{2}, x_2 + \frac{1}{2})$ 
    
    \vspace{1cm}    
  \item $f(x_1, x_2) = e^{-(x_1^2 + x_2^2)}$
  \end{enumerate}
  \vspace{1cm}  

\end{enumerate}

\end{document}
